%% Generated by Sphinx.
\def\sphinxdocclass{report}
\documentclass[letterpaper,10pt,english]{sphinxmanual}
\ifdefined\pdfpxdimen
   \let\sphinxpxdimen\pdfpxdimen\else\newdimen\sphinxpxdimen
\fi \sphinxpxdimen=.75bp\relax

\PassOptionsToPackage{warn}{textcomp}
\usepackage[utf8]{inputenc}
\ifdefined\DeclareUnicodeCharacter
% support both utf8 and utf8x syntaxes
  \ifdefined\DeclareUnicodeCharacterAsOptional
    \def\sphinxDUC#1{\DeclareUnicodeCharacter{"#1}}
  \else
    \let\sphinxDUC\DeclareUnicodeCharacter
  \fi
  \sphinxDUC{00A0}{\nobreakspace}
  \sphinxDUC{2500}{\sphinxunichar{2500}}
  \sphinxDUC{2502}{\sphinxunichar{2502}}
  \sphinxDUC{2514}{\sphinxunichar{2514}}
  \sphinxDUC{251C}{\sphinxunichar{251C}}
  \sphinxDUC{2572}{\textbackslash}
\fi
\usepackage{cmap}
\usepackage[T1]{fontenc}
\usepackage{amsmath,amssymb,amstext}
\usepackage{babel}



\usepackage{times}
\expandafter\ifx\csname T@LGR\endcsname\relax
\else
% LGR was declared as font encoding
  \substitutefont{LGR}{\rmdefault}{cmr}
  \substitutefont{LGR}{\sfdefault}{cmss}
  \substitutefont{LGR}{\ttdefault}{cmtt}
\fi
\expandafter\ifx\csname T@X2\endcsname\relax
  \expandafter\ifx\csname T@T2A\endcsname\relax
  \else
  % T2A was declared as font encoding
    \substitutefont{T2A}{\rmdefault}{cmr}
    \substitutefont{T2A}{\sfdefault}{cmss}
    \substitutefont{T2A}{\ttdefault}{cmtt}
  \fi
\else
% X2 was declared as font encoding
  \substitutefont{X2}{\rmdefault}{cmr}
  \substitutefont{X2}{\sfdefault}{cmss}
  \substitutefont{X2}{\ttdefault}{cmtt}
\fi


\usepackage[Bjarne]{fncychap}
\usepackage{sphinx}

\fvset{fontsize=\small}
\usepackage{geometry}

% Include hyperref last.
\usepackage{hyperref}
% Fix anchor placement for figures with captions.
\usepackage{hypcap}% it must be loaded after hyperref.
% Set up styles of URL: it should be placed after hyperref.
\urlstyle{same}
\addto\captionsenglish{\renewcommand{\contentsname}{Contents:}}

\usepackage{sphinxmessages}
\setcounter{tocdepth}{1}



\title{FullwaveQC}
\date{Aug 21, 2019}
\release{1.0.0}
\author{Deborah Pelacani Cruz}
\newcommand{\sphinxlogo}{\vbox{}}
\renewcommand{\releasename}{Release}
\makeindex
\begin{document}

\pagestyle{empty}
\sphinxmaketitle
\pagestyle{plain}
\sphinxtableofcontents
\pagestyle{normal}
\phantomsection\label{\detokenize{index::doc}}

\index{fullwaveqc.tools (module)@\spxentry{fullwaveqc.tools}\spxextra{module}}\index{Model (class in fullwaveqc.tools)@\spxentry{Model}\spxextra{class in fullwaveqc.tools}}

\begin{fulllineitems}
\phantomsection\label{\detokenize{index:fullwaveqc.tools.Model}}\pysiglinewithargsret{\sphinxbfcode{\sphinxupquote{class }}\sphinxcode{\sphinxupquote{fullwaveqc.tools.}}\sphinxbfcode{\sphinxupquote{Model}}}{\emph{filepath}, \emph{name=None}}{}
Class to store properties of SEG-Y 2D model files

\end{fulllineitems}

\index{SegyData (class in fullwaveqc.tools)@\spxentry{SegyData}\spxextra{class in fullwaveqc.tools}}

\begin{fulllineitems}
\phantomsection\label{\detokenize{index:fullwaveqc.tools.SegyData}}\pysiglinewithargsret{\sphinxbfcode{\sphinxupquote{class }}\sphinxcode{\sphinxupquote{fullwaveqc.tools.}}\sphinxbfcode{\sphinxupquote{SegyData}}}{\emph{filepath}, \emph{name=None}}{}
Class to store properties of SEG-Y data files

\end{fulllineitems}

\index{ampnorm() (in module fullwaveqc.tools)@\spxentry{ampnorm()}\spxextra{in module fullwaveqc.tools}}

\begin{fulllineitems}
\phantomsection\label{\detokenize{index:fullwaveqc.tools.ampnorm}}\pysiglinewithargsret{\sphinxcode{\sphinxupquote{fullwaveqc.tools.}}\sphinxbfcode{\sphinxupquote{ampnorm}}}{\emph{Obs}, \emph{Pred}, \emph{ref\_trace=0}, \emph{verbose=1}}{}
Normalises the amplitude of a predicted dataset with an observed dataset by matching the values of their maximum
amplitudes of a reference trace.
\begin{quote}\begin{description}
\item[{Parameters}] \leavevmode\begin{itemize}
\item {} 
\sphinxstyleliteralstrong{\sphinxupquote{Obs}} ({\hyperref[\detokenize{index:fullwaveqc.tools.SegyData}]{\sphinxcrossref{\sphinxstyleliteralemphasis{\sphinxupquote{fullwaveqc.tools.SegyData}}}}}) \textendash{} Object outputted from fullwaveqc.tools.load function for the observed data

\item {} 
\sphinxstyleliteralstrong{\sphinxupquote{Pred}} ({\hyperref[\detokenize{index:fullwaveqc.tools.SegyData}]{\sphinxcrossref{\sphinxstyleliteralemphasis{\sphinxupquote{fullwaveqc.tools.SegyData}}}}}) \textendash{} object outputted from fullwaveqc.tools.load function for the predicted data.
Requires same number of samples, sampling interval, shots and receiver positions
as Obs

\item {} 
\sphinxstyleliteralstrong{\sphinxupquote{ref\_trace}} (\sphinxstyleliteralemphasis{\sphinxupquote{int}}\sphinxstyleliteralemphasis{\sphinxupquote{, }}\sphinxstyleliteralemphasis{\sphinxupquote{optional}}) \textendash{} Trace number to normalise each shot. For streamer data, this value should be
0 in order to normalise the shot to the first arrival of the first trace. Default 0

\item {} 
\sphinxstyleliteralstrong{\sphinxupquote{verbose}} (\sphinxstyleliteralemphasis{\sphinxupquote{bool}}\sphinxstyleliteralemphasis{\sphinxupquote{, }}\sphinxstyleliteralemphasis{\sphinxupquote{optional}}) \textendash{} Set to true to verbose the steps of this function. Default 1

\end{itemize}

\item[{Returns}] \leavevmode
\sphinxstylestrong{PredNorm} \textendash{} Object outputted from fullwaveqc.tools.load function with the normalise predicted data

\item[{Return type}] \leavevmode
{\hyperref[\detokenize{index:fullwaveqc.tools.SegyData}]{\sphinxcrossref{fullwaveqc.tools.SegyData}}}

\end{description}\end{quote}

\end{fulllineitems}

\index{bandpass() (in module fullwaveqc.tools)@\spxentry{bandpass()}\spxextra{in module fullwaveqc.tools}}

\begin{fulllineitems}
\phantomsection\label{\detokenize{index:fullwaveqc.tools.bandpass}}\pysiglinewithargsret{\sphinxcode{\sphinxupquote{fullwaveqc.tools.}}\sphinxbfcode{\sphinxupquote{bandpass}}}{\emph{trace}, \emph{flow}, \emph{fhigh}, \emph{forder}, \emph{dt}}{}
Band passes a trace using a Butter filter.
\begin{quote}\begin{description}
\item[{Parameters}] \leavevmode\begin{itemize}
\item {} 
\sphinxstyleliteralstrong{\sphinxupquote{trace}} (\sphinxstyleliteralemphasis{\sphinxupquote{numpy.array}}) \textendash{} 1D array containing the signal in time domain

\item {} 
\sphinxstyleliteralstrong{\sphinxupquote{flow}} (\sphinxstyleliteralemphasis{\sphinxupquote{float}}) \textendash{} Low frquency to band pass (Hz)

\item {} 
\sphinxstyleliteralstrong{\sphinxupquote{fhigh}} (\sphinxstyleliteralemphasis{\sphinxupquote{float}}) \textendash{} High frquency to band pass (Hz)

\item {} 
\sphinxstyleliteralstrong{\sphinxupquote{forder}} (\sphinxstyleliteralemphasis{\sphinxupquote{float}}) \textendash{} Order of band pass filter, determines the steepnes of the transition band

\item {} 
\sphinxstyleliteralstrong{\sphinxupquote{dt}} (\sphinxstyleliteralemphasis{\sphinxupquote{float}}) \textendash{} Time sampling of the signal

\end{itemize}

\item[{Returns}] \leavevmode
1D array of same size as trace, with the filtered signal

\item[{Return type}] \leavevmode
filtered, numpy.array

\end{description}\end{quote}

\end{fulllineitems}

\index{ddwi() (in module fullwaveqc.tools)@\spxentry{ddwi()}\spxextra{in module fullwaveqc.tools}}

\begin{fulllineitems}
\phantomsection\label{\detokenize{index:fullwaveqc.tools.ddwi}}\pysiglinewithargsret{\sphinxcode{\sphinxupquote{fullwaveqc.tools.}}\sphinxbfcode{\sphinxupquote{ddwi}}}{\emph{MonObs}, \emph{BaseObs}, \emph{BasePred}, \emph{normalise=True}, \emph{name=None}, \emph{mon\_filepath=None}, \emph{save=False}, \emph{save\_path='./'}, \emph{verbose=1}}{}
Creates the monitor dataset used as the observed dataset for the Double Difference Waveform Inversion.
The DDWI monitor dataset is given by: u\_mon = d\_mon - d\_base + u\_base
Where d\_mon and d\_base are the observed monitor and baseline datasets, respectively, and u\_base is the dataset from
a predicted baseline inversion model.
Note: If data delay has been applied to BaseObs and MonObs, the resulting ddwi dataset will already have the data
delay applied.
\begin{quote}\begin{description}
\item[{Parameters}] \leavevmode\begin{itemize}
\item {} 
\sphinxstyleliteralstrong{\sphinxupquote{MonObs}} ({\hyperref[\detokenize{index:fullwaveqc.tools.SegyData}]{\sphinxcrossref{\sphinxstyleliteralemphasis{\sphinxupquote{fullwaveqc.tools.SegyData}}}}}) \textendash{} Object outputted from fullwaveqc.tools.load function for the true monitor data

\item {} 
\sphinxstyleliteralstrong{\sphinxupquote{BaseObs}} ({\hyperref[\detokenize{index:fullwaveqc.tools.SegyData}]{\sphinxcrossref{\sphinxstyleliteralemphasis{\sphinxupquote{fullwaveqc.tools.SegyData}}}}}) \textendash{} Object outputted from fullwaveqc.tools.load function for the true baseline data.
Requires same number of samples, sampling interval, shots and receiver position
as MonObs

\item {} 
\sphinxstyleliteralstrong{\sphinxupquote{fullwaveqc.tools.SegyData}} (\sphinxstyleliteralemphasis{\sphinxupquote{BasePred}}) \textendash{} Object outputted from fullwaveqc.tools.load function for the predicted baseline data.
Requires same number of samples, sampling interval, shots and receiver position
as MonObs

\item {} 
\sphinxstyleliteralstrong{\sphinxupquote{normalise}} (\sphinxstyleliteralemphasis{\sphinxupquote{bool}}\sphinxstyleliteralemphasis{\sphinxupquote{, }}\sphinxstyleliteralemphasis{\sphinxupquote{optional}}) \textendash{} If true will normalise the amplitude of the predicted baseline dataset to match
the first trace of the observed dataset. Does so shot by shot. Default True

\item {} 
\sphinxstyleliteralstrong{\sphinxupquote{name}} (\sphinxstyleliteralemphasis{\sphinxupquote{str}}\sphinxstyleliteralemphasis{\sphinxupquote{, }}\sphinxstyleliteralemphasis{\sphinxupquote{optional}}) \textendash{} Name to give the DDWI dataset, if None will give the name “DDWI-DATASET”.
Default None

\item {} 
\sphinxstyleliteralstrong{\sphinxupquote{mon\_filepath}} (\sphinxstyleliteralemphasis{\sphinxupquote{str}}\sphinxstyleliteralemphasis{\sphinxupquote{, }}\sphinxstyleliteralemphasis{\sphinxupquote{optional}}) \textendash{} Path to the observed monitor dataset. Must end in .sgy. Required if save option
is set to True. Default None.

\item {} 
\sphinxstyleliteralstrong{\sphinxupquote{save}} (\sphinxstyleliteralemphasis{\sphinxupquote{bool}}\sphinxstyleliteralemphasis{\sphinxupquote{, }}\sphinxstyleliteralemphasis{\sphinxupquote{optional}}) \textendash{} Set to true if the DDWI dataset is to be saved in .sgy format. Default False

\item {} 
\sphinxstyleliteralstrong{\sphinxupquote{save\_path}} (\sphinxstyleliteralemphasis{\sphinxupquote{str}}\sphinxstyleliteralemphasis{\sphinxupquote{, }}\sphinxstyleliteralemphasis{\sphinxupquote{optional}}) \textendash{} Path to folder where DDWI must be saved. Default “./”

\item {} 
\sphinxstyleliteralstrong{\sphinxupquote{verbose}} (\sphinxstyleliteralemphasis{\sphinxupquote{bool}}\sphinxstyleliteralemphasis{\sphinxupquote{, }}\sphinxstyleliteralemphasis{\sphinxupquote{optional}}) \textendash{} Set to true in order to verbose the steps of this function. Default 1

\end{itemize}

\item[{Returns}] \leavevmode
\sphinxstylestrong{MonDiff} \textendash{} Object outputted from fullwaveqc.tools.load function with the DDWI monitor data

\item[{Return type}] \leavevmode
{\hyperref[\detokenize{index:fullwaveqc.tools.SegyData}]{\sphinxcrossref{fullwaveqc.tools.SegyData}}}

\end{description}\end{quote}
\subsubsection*{Notes}

mon\_filepath must be given if save is set to True. Saving occurs by copying and modifying
the monitor SEG-Y headers and traces.

\end{fulllineitems}

\index{load() (in module fullwaveqc.tools)@\spxentry{load()}\spxextra{in module fullwaveqc.tools}}

\begin{fulllineitems}
\phantomsection\label{\detokenize{index:fullwaveqc.tools.load}}\pysiglinewithargsret{\sphinxcode{\sphinxupquote{fullwaveqc.tools.}}\sphinxbfcode{\sphinxupquote{load}}}{\emph{filepath}, \emph{model}, \emph{name=None}, \emph{scale=1}, \emph{verbose=1}}{}
Loads a SEG-Y file into a SegyData or Model class, used for the other functionalities of the fullwaveqc package.
Accepts 2D segy files and of the same sampling interval for all shots and segy model files with square cells.
It is HIGHLY recommended that the attributes of this function are checked manually after loading, as different
SEG-Y header formats are likely to be loaded incorrectly. This function is adapted to the SEGY header format of
the outputs of Fullwave3D.
\begin{quote}\begin{description}
\item[{Parameters}] \leavevmode\begin{itemize}
\item {} 
\sphinxstyleliteralstrong{\sphinxupquote{filepath}} (\sphinxstyleliteralemphasis{\sphinxupquote{str}}) \textendash{} Path to segy file. Must end in .sgy

\item {} 
\sphinxstyleliteralstrong{\sphinxupquote{model}} (\sphinxstyleliteralemphasis{\sphinxupquote{bool}}) \textendash{} Set true to load a Model and false to load a SegyData object

\item {} 
\sphinxstyleliteralstrong{\sphinxupquote{name}} (\sphinxstyleliteralemphasis{\sphinxupquote{str}}\sphinxstyleliteralemphasis{\sphinxupquote{, }}\sphinxstyleliteralemphasis{\sphinxupquote{optional}}) \textendash{} Name to give the data/model object. If None is given, name is inferred from .sgy file. Default is None

\item {} 
\sphinxstyleliteralstrong{\sphinxupquote{scale}} (\sphinxstyleliteralemphasis{\sphinxupquote{int}}\sphinxstyleliteralemphasis{\sphinxupquote{, }}\sphinxstyleliteralemphasis{\sphinxupquote{optional}}) \textendash{} Value to multiply the data content of the files. Default 1

\item {} 
\sphinxstyleliteralstrong{\sphinxupquote{verbose}} (\sphinxstyleliteralemphasis{\sphinxupquote{bool}}\sphinxstyleliteralemphasis{\sphinxupquote{, }}\sphinxstyleliteralemphasis{\sphinxupquote{optional}}) \textendash{} Set to true in order to verbose the steps of this loading function. Default True

\end{itemize}

\item[{Returns}] \leavevmode


\item[{Return type}] \leavevmode
None

\end{description}\end{quote}

\end{fulllineitems}

\index{rm\_empty\_traces() (in module fullwaveqc.tools)@\spxentry{rm\_empty\_traces()}\spxextra{in module fullwaveqc.tools}}

\begin{fulllineitems}
\phantomsection\label{\detokenize{index:fullwaveqc.tools.rm_empty_traces}}\pysiglinewithargsret{\sphinxcode{\sphinxupquote{fullwaveqc.tools.}}\sphinxbfcode{\sphinxupquote{rm\_empty\_traces}}}{\emph{filename}, \emph{scale=1}, \emph{verbose=1}}{}
Removes all-zero traces of a SEG-Y file and saves the copy of the clean file in the same directory
\begin{quote}\begin{description}
\item[{Parameters}] \leavevmode\begin{itemize}
\item {} 
\sphinxstyleliteralstrong{\sphinxupquote{filename}} (\sphinxstyleliteralemphasis{\sphinxupquote{str}}) \textendash{} Path to SEG-Y file, must end in .sgy

\item {} 
\sphinxstyleliteralstrong{\sphinxupquote{scale}} (\sphinxstyleliteralemphasis{\sphinxupquote{int}}\sphinxstyleliteralemphasis{\sphinxupquote{, }}\sphinxstyleliteralemphasis{\sphinxupquote{optional}}) \textendash{} Value to multiply the data content of the files. Default 1

\item {} 
\sphinxstyleliteralstrong{\sphinxupquote{verbose}} (\sphinxstyleliteralemphasis{\sphinxupquote{bool}}\sphinxstyleliteralemphasis{\sphinxupquote{, }}\sphinxstyleliteralemphasis{\sphinxupquote{optional}}) \textendash{} Set to true in order to verbose the steps of this loading function. Default True

\end{itemize}

\item[{Returns}] \leavevmode


\item[{Return type}] \leavevmode
None

\end{description}\end{quote}

\end{fulllineitems}

\index{set\_verbose() (in module fullwaveqc.tools)@\spxentry{set\_verbose()}\spxextra{in module fullwaveqc.tools}}

\begin{fulllineitems}
\phantomsection\label{\detokenize{index:fullwaveqc.tools.set_verbose}}\pysiglinewithargsret{\sphinxcode{\sphinxupquote{fullwaveqc.tools.}}\sphinxbfcode{\sphinxupquote{set\_verbose}}}{\emph{verbose}}{}
Function to set the verbose print level
:param verbose: Verbose level, set to true to display verbose
:type verbose: bool
\begin{quote}\begin{description}
\item[{Returns}] \leavevmode
\sphinxstylestrong{verbose\_print(message)} \textendash{} Function to display verbose message in console

\item[{Return type}] \leavevmode
lamda function

\end{description}\end{quote}

\end{fulllineitems}

\index{smooth\_model() (in module fullwaveqc.tools)@\spxentry{smooth\_model()}\spxextra{in module fullwaveqc.tools}}

\begin{fulllineitems}
\phantomsection\label{\detokenize{index:fullwaveqc.tools.smooth_model}}\pysiglinewithargsret{\sphinxcode{\sphinxupquote{fullwaveqc.tools.}}\sphinxbfcode{\sphinxupquote{smooth\_model}}}{\emph{Model, strength={[}1, 1{]}, w={[}None, None, None, None{]}, slowness=True, name=None, save=False, save\_path='./', verbose=1}}{}
Smoothes a model using scipy’s gaussian filter in “reflect” mode.
\begin{quote}\begin{description}
\item[{Parameters}] \leavevmode\begin{itemize}
\item {} 
\sphinxstyleliteralstrong{\sphinxupquote{Model}} ({\hyperref[\detokenize{index:fullwaveqc.tools.Model}]{\sphinxcrossref{\sphinxstyleliteralemphasis{\sphinxupquote{fullwaveqc.tools.Model}}}}}) \textendash{} Object outputted from fullwaveqc.tools.load function for a model. Must have a valid filepath attribute.

\item {} 
\sphinxstyleliteralstrong{\sphinxupquote{strength}} (\sphinxstyleliteralemphasis{\sphinxupquote{list}}\sphinxstyleliteralemphasis{\sphinxupquote{{[}}}\sphinxstyleliteralemphasis{\sphinxupquote{floats}}\sphinxstyleliteralemphasis{\sphinxupquote{{]}}}\sphinxstyleliteralemphasis{\sphinxupquote{, }}\sphinxstyleliteralemphasis{\sphinxupquote{optional}}) \textendash{} {[}horizontal\_smoothing\_factor, vertical\_smoothing\_factor{]}. Default {[}1,1{]}

\item {} 
\sphinxstyleliteralstrong{\sphinxupquote{w}} (\sphinxstyleliteralemphasis{\sphinxupquote{list}}\sphinxstyleliteralemphasis{\sphinxupquote{{[}}}\sphinxstyleliteralemphasis{\sphinxupquote{ints}}\sphinxstyleliteralemphasis{\sphinxupquote{{]}}}\sphinxstyleliteralemphasis{\sphinxupquote{, }}\sphinxstyleliteralemphasis{\sphinxupquote{optional}}) \textendash{} Windowing of the smoothing. Smoothing will be applied to the internal window created by the rectangle determined
by {[}cells\_from\_top, cells\_from bottom, cells\_from\_left, cells\_from\_right{]}. If None is given, then either the be
ginning or end will of the model will be used. Default {[}None, None, None, None{]}

\item {} 
\sphinxstyleliteralstrong{\sphinxupquote{slowness}} (\sphinxstyleliteralemphasis{\sphinxupquote{bool}}\sphinxstyleliteralemphasis{\sphinxupquote{, }}\sphinxstyleliteralemphasis{\sphinxupquote{optional}}) \textendash{} If True will smooth the slowness instead of velocities. Slowness defined
as the reciprocal value of velocity and is commonly preferred in smoothing. Default True

\item {} 
\sphinxstyleliteralstrong{\sphinxupquote{name}} (\sphinxstyleliteralemphasis{\sphinxupquote{str}}\sphinxstyleliteralemphasis{\sphinxupquote{, }}\sphinxstyleliteralemphasis{\sphinxupquote{optional}}) \textendash{} name of the new smoothed model. If None it will be inferred from Model.
Default None

\item {} 
\sphinxstyleliteralstrong{\sphinxupquote{save}} (\sphinxstyleliteralemphasis{\sphinxupquote{bool}}\sphinxstyleliteralemphasis{\sphinxupquote{, }}\sphinxstyleliteralemphasis{\sphinxupquote{optional}}) \textendash{} Set to true in order to save the model in .sgy format. Requires that Model
has a valid filepath attribute. Default False

\item {} 
\sphinxstyleliteralstrong{\sphinxupquote{save\_path}} (\sphinxstyleliteralemphasis{\sphinxupquote{str}}\sphinxstyleliteralemphasis{\sphinxupquote{, }}\sphinxstyleliteralemphasis{\sphinxupquote{optional}}) \textendash{} Path to save the .sgy smoothed model. Default “./”

\item {} 
\sphinxstyleliteralstrong{\sphinxupquote{verbose}} (\sphinxstyleliteralemphasis{\sphinxupquote{bool}}\sphinxstyleliteralemphasis{\sphinxupquote{, }}\sphinxstyleliteralemphasis{\sphinxupquote{optional}}) \textendash{} 

\end{itemize}

\item[{Returns}] \leavevmode
\sphinxstylestrong{SmoothModel} \textendash{} Object as outputted from fullwaveqc.tools.load function containing the smoothed model

\item[{Return type}] \leavevmode
{\hyperref[\detokenize{index:fullwaveqc.tools.Model}]{\sphinxcrossref{fullwaveqc.tools.Model}}}

\end{description}\end{quote}

\end{fulllineitems}

\phantomsection\label{\detokenize{index:module-fullwaveqc.geom}}\index{fullwaveqc.geom (module)@\spxentry{fullwaveqc.geom}\spxextra{module}}\index{boundarycalc() (in module fullwaveqc.geom)@\spxentry{boundarycalc()}\spxextra{in module fullwaveqc.geom}}

\begin{fulllineitems}
\phantomsection\label{\detokenize{index:fullwaveqc.geom.boundarycalc}}\pysiglinewithargsret{\sphinxcode{\sphinxupquote{fullwaveqc.geom.}}\sphinxbfcode{\sphinxupquote{boundarycalc}}}{\emph{d}, \emph{dx}, \emph{fmin}, \emph{vmax}}{}
Calculates the number of absorbing boundaries and number of cells from the source recommended for
a Fullwave run without reflections and interferences. Number of absorbing boundaries computed as
to guarantee it covers two of the largest wavelengths. Distance from source computed to guarantee
one Fresnel zone between the source and the first receiver. The total number of padding cells above
the source should therefore be nabsorb + ndist.
\begin{quote}\begin{description}
\item[{Parameters}] \leavevmode\begin{itemize}
\item {} 
\sphinxstyleliteralstrong{\sphinxupquote{d}} \textendash{} (float) distance between source and first receiver in units of distance (meters)

\item {} 
\sphinxstyleliteralstrong{\sphinxupquote{dx}} \textendash{} (float) size of model cell in units of distance (meters)

\item {} 
\sphinxstyleliteralstrong{\sphinxupquote{fmin}} \textendash{} (float) minimum frequency present in the wavelet (Hz)

\item {} 
\sphinxstyleliteralstrong{\sphinxupquote{vmax}} \textendash{} (float) maximum P-wave velocity expected from the models (m/s)

\end{itemize}

\item[{Returns}] \leavevmode
nabsorb (int)   number of absorbing boundary cells recommended for the specific problem

\item[{Returns}] \leavevmode
ndist   (int)   number of padded cells from the source to the start of the absorbing boundaries

\end{description}\end{quote}

\end{fulllineitems}

\index{surveygeom() (in module fullwaveqc.geom)@\spxentry{surveygeom()}\spxextra{in module fullwaveqc.geom}}

\begin{fulllineitems}
\phantomsection\label{\detokenize{index:fullwaveqc.geom.surveygeom}}\pysiglinewithargsret{\sphinxcode{\sphinxupquote{fullwaveqc.geom.}}\sphinxbfcode{\sphinxupquote{surveygeom}}}{\emph{rcvgeopath}, \emph{srcgeopath}, \emph{src\_list={[}{]}}, \emph{plot=False}, \emph{verbose=0}}{}
Retrieves and plots the in-line positions of the survey array as understood from SegyPrep.
\begin{quote}\begin{description}
\item[{Parameters}] \leavevmode\begin{itemize}
\item {} 
\sphinxstyleliteralstrong{\sphinxupquote{rcvgeopath}} \textendash{} (str)  path to file \textless{}PROJECT\_NAME\textgreater{}-Receivers.geo generated by SegyPrep

\item {} 
\sphinxstyleliteralstrong{\sphinxupquote{srcgeopath}} \textendash{} (str)  path to file \textless{}PROJECT\_NAME\textgreater{}-Sources.geo generated by SegyPrep

\item {} 
\sphinxstyleliteralstrong{\sphinxupquote{src\_list}} \textendash{} (list) list of source/shot numbers to retrieve and plot. Empty list returns
all positions. Default: {[}{]}

\item {} 
\sphinxstyleliteralstrong{\sphinxupquote{plot}} \textendash{} (bool) If true will plot the shots and receivers of src\_list. Default:False

\item {} 
\sphinxstyleliteralstrong{\sphinxupquote{verbose}} \textendash{} (bool) If true will print to console the steps of this function. Default: False

\end{itemize}

\item[{Returns}] \leavevmode
src\_ret     (list) list of all in-line positions from the sources in src\_list

\item[{Returns}] \leavevmode
rcv\_ret     (list) list of numpy.arrays, for each source in src\_list, with its array of in-line
receiver positions

\end{description}\end{quote}

\end{fulllineitems}

\phantomsection\label{\detokenize{index:module-fullwaveqc.visual}}\index{fullwaveqc.visual (module)@\spxentry{fullwaveqc.visual}\spxextra{module}}\index{amplitude() (in module fullwaveqc.visual)@\spxentry{amplitude()}\spxextra{in module fullwaveqc.visual}}

\begin{fulllineitems}
\phantomsection\label{\detokenize{index:fullwaveqc.visual.amplitude}}\pysiglinewithargsret{\sphinxcode{\sphinxupquote{fullwaveqc.visual.}}\sphinxbfcode{\sphinxupquote{amplitude}}}{\emph{SegyData}, \emph{shot=1}, \emph{cap=0.0}, \emph{levels=50}, \emph{vmin=None}, \emph{vmax=None}, \emph{cmap=\textless{}matplotlib.colors.LinearSegmentedColormap object\textgreater{}}, \emph{xstart=0}, \emph{xend=None}, \emph{wstart=0}, \emph{wend=None}, \emph{save=False}, \emph{save\_path='./'}}{}
Plots the amplitude map of a SegyData object in timesamples (dt units) vs receiver index.
Uses properties of matplotlib.contourf.
\begin{quote}\begin{description}
\item[{Parameters}] \leavevmode\begin{itemize}
\item {} 
\sphinxstyleliteralstrong{\sphinxupquote{SegyData}} \textendash{} 

\item {} 
\sphinxstyleliteralstrong{\sphinxupquote{shot}} \textendash{} 

\item {} 
\sphinxstyleliteralstrong{\sphinxupquote{cap}} \textendash{} 

\item {} 
\sphinxstyleliteralstrong{\sphinxupquote{levels}} \textendash{} 

\item {} 
\sphinxstyleliteralstrong{\sphinxupquote{vmin}} \textendash{} 

\item {} 
\sphinxstyleliteralstrong{\sphinxupquote{vmax}} \textendash{} 

\item {} 
\sphinxstyleliteralstrong{\sphinxupquote{cmap}} \textendash{} 

\item {} 
\sphinxstyleliteralstrong{\sphinxupquote{xstart}} \textendash{} 

\item {} 
\sphinxstyleliteralstrong{\sphinxupquote{xend}} \textendash{} 

\item {} 
\sphinxstyleliteralstrong{\sphinxupquote{wstart}} \textendash{} 

\item {} 
\sphinxstyleliteralstrong{\sphinxupquote{wend}} \textendash{} 

\item {} 
\sphinxstyleliteralstrong{\sphinxupquote{save}} \textendash{} 

\item {} 
\sphinxstyleliteralstrong{\sphinxupquote{save\_path}} \textendash{} 

\end{itemize}

\end{description}\end{quote}

\end{fulllineitems}

\index{animateinv() (in module fullwaveqc.visual)@\spxentry{animateinv()}\spxextra{in module fullwaveqc.visual}}

\begin{fulllineitems}
\phantomsection\label{\detokenize{index:fullwaveqc.visual.animateinv}}\pysiglinewithargsret{\sphinxcode{\sphinxupquote{fullwaveqc.visual.}}\sphinxbfcode{\sphinxupquote{animateinv}}}{\emph{it\_max}, \emph{path}, \emph{project\_name}, \emph{vmin=None}, \emph{vmax=None}, \emph{cmap=\textless{}matplotlib.colors.LinearSegmentedColormap object\textgreater{}}, \emph{save=False}, \emph{save\_path='./'}, \emph{verbose=0}}{}
Animates the evolution of an inversion progression outputted by Fullwave3D. Files searched for must be named:
\textless{}project\_name\textgreater{}-CPxxxxx-Vp.sgy and be all located in the same path folder.
\begin{quote}\begin{description}
\item[{Parameters}] \leavevmode\begin{itemize}
\item {} 
\sphinxstyleliteralstrong{\sphinxupquote{it\_max}} \textendash{} (int)        max amount of iterations to be analysed

\item {} 
\sphinxstyleliteralstrong{\sphinxupquote{path}} \textendash{} (str)        path to folder where .sgy model files are located

\item {} 
\sphinxstyleliteralstrong{\sphinxupquote{project\_name}} \textendash{} (str)        name of the project \textendash{} must match exactly

\item {} 
\sphinxstyleliteralstrong{\sphinxupquote{vmin}} \textendash{} (float)      min val of color plot scale. Default None.

\item {} 
\sphinxstyleliteralstrong{\sphinxupquote{vmax}} \textendash{} (float)      max val of color plot scale. Default None.

\item {} 
\sphinxstyleliteralstrong{\sphinxupquote{cmap}} \textendash{} (plt.cm)     cmap object. Default plt.cm.jet

\item {} 
\sphinxstyleliteralstrong{\sphinxupquote{save}} \textendash{} (bool)       set to true in order to save the animation

\item {} 
\sphinxstyleliteralstrong{\sphinxupquote{save\_path}} \textendash{} (str)        path to save the animation. Default “./”

\item {} 
\sphinxstyleliteralstrong{\sphinxupquote{verbose}} \textendash{} (bool)       set to true to verbose the steps of this function

\end{itemize}

\item[{Returns}] \leavevmode
None

\end{description}\end{quote}

\end{fulllineitems}

\index{interamp() (in module fullwaveqc.visual)@\spxentry{interamp()}\spxextra{in module fullwaveqc.visual}}

\begin{fulllineitems}
\phantomsection\label{\detokenize{index:fullwaveqc.visual.interamp}}\pysiglinewithargsret{\sphinxcode{\sphinxupquote{fullwaveqc.visual.}}\sphinxbfcode{\sphinxupquote{interamp}}}{\emph{SegyData1}, \emph{SegyData2}, \emph{shot=1}, \emph{shot2=None}, \emph{n\_blocks=2}, \emph{cap=0.0}, \emph{levels=50}, \emph{vmin=None}, \emph{vmax=None}, \emph{cmap=\textless{}matplotlib.colors.LinearSegmentedColormap object\textgreater{}}, \emph{wstart=0}, \emph{wend=None}, \emph{xstart=0}, \emph{xend=None}, \emph{save=False}, \emph{save\_path='./'}}{}
Plots the interleaving amplitude map of a SegyData1 and SegyData2 objects in timesamples vs receiver index.
Uses properties of matplotlib.contourf.
\begin{quote}\begin{description}
\item[{Parameters}] \leavevmode\begin{itemize}
\item {} 
\sphinxstyleliteralstrong{\sphinxupquote{SegyData1}} \textendash{} (SegyData)   object outputted from fullwaveqc.tools.load function for the segy data

\item {} 
\sphinxstyleliteralstrong{\sphinxupquote{SegyData2}} \textendash{} (SegyData)   object outputted from fullwaveqc.tools.load function for the segy data
Requires same number of samples, sampling interval, shots and receiver
positions as SegyData1

\item {} 
\sphinxstyleliteralstrong{\sphinxupquote{shot}} \textendash{} (int)        shot number to visualise SegyData1. Default 1

\item {} 
\sphinxstyleliteralstrong{\sphinxupquote{shot2}} \textendash{} (int)        shot number to visualise SegyData2. Count starts from 1. If None, will .
be read the same as ‘shot’. Default None

\item {} 
\sphinxstyleliteralstrong{\sphinxupquote{n\_blocks}} \textendash{} (int)        Number of total blocks in the interleaving space

\item {} 
\sphinxstyleliteralstrong{\sphinxupquote{cap}} \textendash{} (float)      absolute value of amplitudes to cap, shown as 0. Default 0.

\item {} 
\sphinxstyleliteralstrong{\sphinxupquote{levels}} \textendash{} (int)        amount of levels in the contour plot. Default 100

\item {} 
\sphinxstyleliteralstrong{\sphinxupquote{vmin}} \textendash{} (float)      min val of color plot scale. Default None.

\item {} 
\sphinxstyleliteralstrong{\sphinxupquote{vmax}} \textendash{} (float)      max val of color plot scale. Default None.

\item {} 
\sphinxstyleliteralstrong{\sphinxupquote{cmap}} \textendash{} (plt.cm)     cmap object. Default plt.cm.seismic

\item {} 
\sphinxstyleliteralstrong{\sphinxupquote{wstart}} \textendash{} (int)        first timesample in time units to plot. Default 0

\item {} 
\sphinxstyleliteralstrong{\sphinxupquote{wend}} \textendash{} (int)        last timesample in time units to plot. Default None

\item {} 
\sphinxstyleliteralstrong{\sphinxupquote{save}} \textendash{} (bool)       set to true in order to save the plot in png 300dpi. Default False

\item {} 
\sphinxstyleliteralstrong{\sphinxupquote{save\_path}} \textendash{} (str)        path to save the plot. Default “./”

\end{itemize}

\item[{Returns}] \leavevmode
None

\item[{Returns}] \leavevmode


\end{description}\end{quote}

\end{fulllineitems}

\index{interwiggle() (in module fullwaveqc.visual)@\spxentry{interwiggle()}\spxextra{in module fullwaveqc.visual}}

\begin{fulllineitems}
\phantomsection\label{\detokenize{index:fullwaveqc.visual.interwiggle}}\pysiglinewithargsret{\sphinxcode{\sphinxupquote{fullwaveqc.visual.}}\sphinxbfcode{\sphinxupquote{interwiggle}}}{\emph{SegyData1}, \emph{SegyData2}, \emph{shot=1}, \emph{shot2=None}, \emph{overlay=0}, \emph{scale=5}, \emph{skip\_trace=20}, \emph{skip\_time=0}, \emph{delay\_samples=0}, \emph{wstart=0}, \emph{wend=None}, \emph{xstart=0}, \emph{xend=None}, \emph{label1='SegyData1'}, \emph{label2='SegyData2'}, \emph{save=False}, \emph{save\_path='./'}}{}
Plots the interleaving wiggle trace map of a SegyData1 and SegyData2 objects in timesamples vs receiver index.
\begin{quote}\begin{description}
\item[{Parameters}] \leavevmode\begin{itemize}
\item {} 
\sphinxstyleliteralstrong{\sphinxupquote{SegyData1}} \textendash{} (SegyData)   object outputted from fullwaveqc.tools.load function for the segy data

\item {} 
\sphinxstyleliteralstrong{\sphinxupquote{SegyData2}} \textendash{} (SegyData)   object outputted from fullwaveqc.tools.load function for the segy data

\item {} 
\sphinxstyleliteralstrong{\sphinxupquote{shot}} \textendash{} (int)        shot number to visualise SegyData1. Default 1

\item {} 
\sphinxstyleliteralstrong{\sphinxupquote{shot2}} \textendash{} (int)        shot number to visualise SegyData2. Count starts from 1. If None, will
be read the same as ‘shot’. Default None

\item {} 
\sphinxstyleliteralstrong{\sphinxupquote{overlay}} \textendash{} (int)        If +ve, traces will be overlaid in different colours instead of being
displayed side by side. Overlay=1 overlays with all filled colors, overlay=2
overlays with SegyData2 as filled colors and SegyData1 as conotur. Default 0

\item {} 
\sphinxstyleliteralstrong{\sphinxupquote{scale}} \textendash{} (float)      value to scale the amplitude of the wiggles for visualisation only. Default 1

\item {} 
\sphinxstyleliteralstrong{\sphinxupquote{skip\_trace}} \textendash{} (int)        Number of traces to skip. Default 0.

\item {} 
\sphinxstyleliteralstrong{\sphinxupquote{skip\_time}} \textendash{} (int)        Number of time samples to skip. Default 0

\item {} 
\sphinxstyleliteralstrong{\sphinxupquote{xstart}} \textendash{} (int)        first receiver index to plot. Default 0.

\item {} 
\sphinxstyleliteralstrong{\sphinxupquote{xend}} \textendash{} (int)        last receiver index to plot. Default None

\item {} 
\sphinxstyleliteralstrong{\sphinxupquote{wstart}} \textendash{} (int)        first timesample in time units to plot. Default 0

\item {} 
\sphinxstyleliteralstrong{\sphinxupquote{wend}} \textendash{} (int)        last timesample in time units to plot. Default None

\item {} 
\sphinxstyleliteralstrong{\sphinxupquote{delay\_samples}} \textendash{} (int)        number of time samples to delay the signal. Will pad the signal with 0s at
the beginning. Default 0

\item {} 
\sphinxstyleliteralstrong{\sphinxupquote{label1}} \textendash{} (str)        Label for SegyData1. Default SegyData1

\item {} 
\sphinxstyleliteralstrong{\sphinxupquote{label2}} \textendash{} (str)        Label for SegyData2. Default SegyData2

\item {} 
\sphinxstyleliteralstrong{\sphinxupquote{save}} \textendash{} (bool)       set to true in order to save the plot in png 300dpi. Default False

\item {} 
\sphinxstyleliteralstrong{\sphinxupquote{save\_path}} \textendash{} (str)        path to save the plot. Default “./”

\end{itemize}

\item[{Returns}] \leavevmode
None

\end{description}\end{quote}

\end{fulllineitems}

\index{vpmodel() (in module fullwaveqc.visual)@\spxentry{vpmodel()}\spxextra{in module fullwaveqc.visual}}

\begin{fulllineitems}
\phantomsection\label{\detokenize{index:fullwaveqc.visual.vpmodel}}\pysiglinewithargsret{\sphinxcode{\sphinxupquote{fullwaveqc.visual.}}\sphinxbfcode{\sphinxupquote{vpmodel}}}{\emph{Model}, \emph{cap=0.0}, \emph{levels=200}, \emph{vmin=None}, \emph{vmax=None}, \emph{cmap=\textless{}matplotlib.colors.LinearSegmentedColormap object\textgreater{}}, \emph{units='m/s'}, \emph{save=False}, \emph{save\_path='./'}}{}
Plots the P-velocity amplitude map of a Model object in depth and lateral offset.
Uses properties of matplotlib.contourf.
\begin{quote}\begin{description}
\item[{Parameters}] \leavevmode\begin{itemize}
\item {} 
\sphinxstyleliteralstrong{\sphinxupquote{Model}} \textendash{} (Model)      object outputted from fullwaveqc.tools.load function for a model

\item {} 
\sphinxstyleliteralstrong{\sphinxupquote{cap}} \textendash{} (float)      absolute value of P-wave values to cap, below this all are shown as 0. Default 0.

\item {} 
\sphinxstyleliteralstrong{\sphinxupquote{levels}} \textendash{} (int)        amount of levels in the contour plot. Default 200

\item {} 
\sphinxstyleliteralstrong{\sphinxupquote{vmin}} \textendash{} (float)      min val of color plot scale. Default None.

\item {} 
\sphinxstyleliteralstrong{\sphinxupquote{vmax}} \textendash{} (float)      max val of color plot scale. Default None.

\item {} 
\sphinxstyleliteralstrong{\sphinxupquote{cmap}} \textendash{} (plt.cm)     cmap object. Default plt.cm.jet

\item {} 
\sphinxstyleliteralstrong{\sphinxupquote{units}} \textendash{} (str)        units of amplitudes to show in colorbar. Default m/s

\item {} 
\sphinxstyleliteralstrong{\sphinxupquote{save}} \textendash{} (bool)       set to true in order to save the plot in png 300dpi. Default False

\item {} 
\sphinxstyleliteralstrong{\sphinxupquote{save\_path}} \textendash{} (str)        path to save the plot. Default “./”

\end{itemize}

\item[{Returns}] \leavevmode
None

\end{description}\end{quote}

\end{fulllineitems}

\index{vpwell() (in module fullwaveqc.visual)@\spxentry{vpwell()}\spxextra{in module fullwaveqc.visual}}

\begin{fulllineitems}
\phantomsection\label{\detokenize{index:fullwaveqc.visual.vpwell}}\pysiglinewithargsret{\sphinxcode{\sphinxupquote{fullwaveqc.visual.}}\sphinxbfcode{\sphinxupquote{vpwell}}}{\emph{Model}, \emph{pos\_x}, \emph{TrueModel=None}, \emph{plot=True}}{}
Retrieves the Vp well profile from a Model at specified locations
\begin{quote}\begin{description}
\item[{Parameters}] \leavevmode\begin{itemize}
\item {} 
\sphinxstyleliteralstrong{\sphinxupquote{Model}} \textendash{} (Model)        object outputted from fullwaveqc.tools.load function for a model

\item {} 
\sphinxstyleliteralstrong{\sphinxupquote{pos\_x}} \textendash{} (list of ints) lateral distances to which retrieve the well data

\item {} 
\sphinxstyleliteralstrong{\sphinxupquote{TrueModel}} \textendash{} (Model)        object outputted from fullwaveqc.tools.load function for the true model.
Default None

\item {} 
\sphinxstyleliteralstrong{\sphinxupquote{plot}} \textendash{} (bool)         Plots the profiles when set to true.

\end{itemize}

\item[{Returns}] \leavevmode
wells, true\_wells, rmses (np.arrays) with well values and root mean square errors as compared to a true
model, or simply returns well if no TrueModel is given.

\end{description}\end{quote}

\end{fulllineitems}

\index{wiggle() (in module fullwaveqc.visual)@\spxentry{wiggle()}\spxextra{in module fullwaveqc.visual}}

\begin{fulllineitems}
\phantomsection\label{\detokenize{index:fullwaveqc.visual.wiggle}}\pysiglinewithargsret{\sphinxcode{\sphinxupquote{fullwaveqc.visual.}}\sphinxbfcode{\sphinxupquote{wiggle}}}{\emph{SegyData}, \emph{shot=1}, \emph{scale=5}, \emph{skip\_trace=0}, \emph{skip\_time=0}, \emph{wstart=0.0}, \emph{wend=None}, \emph{xstart=0}, \emph{xend=None}, \emph{delay\_samples=0}, \emph{save=False}, \emph{save\_path='./'}}{}~\begin{quote}

Plots the wiggle trace map of a SegyData object in timesamples vs receiver index.
\end{quote}
\begin{quote}\begin{description}
\item[{Parameters}] \leavevmode\begin{itemize}
\item {} 
\sphinxstyleliteralstrong{\sphinxupquote{SegyData}} \textendash{} (SegyData)   object outputted from fullwaveqc.tools.load function for the segy data

\item {} 
\sphinxstyleliteralstrong{\sphinxupquote{shot}} \textendash{} (int)        shot number to visualise. Count starts from 1. Default 1

\item {} 
\sphinxstyleliteralstrong{\sphinxupquote{scale}} \textendash{} (float)      value to scale the amplitude of the wiggles for visualisation only. Default 1

\item {} 
\sphinxstyleliteralstrong{\sphinxupquote{skip\_trace}} \textendash{} (int)        Number of traces to skip. Default 0.

\item {} 
\sphinxstyleliteralstrong{\sphinxupquote{skip\_time}} \textendash{} (int)        Number of time samples to skip. Default 0

\item {} 
\sphinxstyleliteralstrong{\sphinxupquote{xstart}} \textendash{} (int)        first receiver index to plot. Default 0.

\item {} 
\sphinxstyleliteralstrong{\sphinxupquote{xend}} \textendash{} (int)        last receiver index to plot. Default None

\item {} 
\sphinxstyleliteralstrong{\sphinxupquote{wstart}} \textendash{} (int)        first timesample in time units to plot. Default 0

\item {} 
\sphinxstyleliteralstrong{\sphinxupquote{wend}} \textendash{} (int)        last timesample in time units to plot. Default None

\item {} 
\sphinxstyleliteralstrong{\sphinxupquote{delay\_samples}} \textendash{} (int)        number of time samples to delay the signal. Will pad the signal with 0s at the
beginning. Default 0.

\item {} 
\sphinxstyleliteralstrong{\sphinxupquote{save}} \textendash{} (bool)       set to true in order to save the plot in png 300dpi. Default False

\item {} 
\sphinxstyleliteralstrong{\sphinxupquote{save\_path}} \textendash{} (str)        path to save the plot. Default “./”

\end{itemize}

\item[{Returns}] \leavevmode
None

\end{description}\end{quote}

\end{fulllineitems}

\phantomsection\label{\detokenize{index:module-fullwaveqc.siganalysis}}\index{fullwaveqc.siganalysis (module)@\spxentry{fullwaveqc.siganalysis}\spxextra{module}}\index{closest2pow() (in module fullwaveqc.siganalysis)@\spxentry{closest2pow()}\spxextra{in module fullwaveqc.siganalysis}}

\begin{fulllineitems}
\phantomsection\label{\detokenize{index:fullwaveqc.siganalysis.closest2pow}}\pysiglinewithargsret{\sphinxcode{\sphinxupquote{fullwaveqc.siganalysis.}}\sphinxbfcode{\sphinxupquote{closest2pow}}}{\emph{n}}{}
Finds the first integer greater or equal to n that is a power of 2

:param   n          (float) any positive number
:return:  n2pow      (int) the first integer greater or equal to n that is an exact power of two
:raises  ValueError if n is less or equal to zero

\end{fulllineitems}

\index{dataspec() (in module fullwaveqc.siganalysis)@\spxentry{dataspec()}\spxextra{in module fullwaveqc.siganalysis}}

\begin{fulllineitems}
\phantomsection\label{\detokenize{index:fullwaveqc.siganalysis.dataspec}}\pysiglinewithargsret{\sphinxcode{\sphinxupquote{fullwaveqc.siganalysis.}}\sphinxbfcode{\sphinxupquote{dataspec}}}{\emph{SegyData}, \emph{ms=True}, \emph{shot=1}, \emph{fmax=None}, \emph{fft\_smooth=1}, \emph{plot=False}}{}
Returns and plots the frequency spectrum of a single shot of a dataset. Does so by stacking the frequencies of each
individual trace.
\begin{quote}\begin{description}
\item[{Parameters}] \leavevmode\begin{itemize}
\item {} 
\sphinxstyleliteralstrong{\sphinxupquote{SegyData}} \textendash{} (SegyData)  object outputted from fullwaveqc.tools.load function

\item {} 
\sphinxstyleliteralstrong{\sphinxupquote{ms}} \textendash{} (bool)      Set to true if sampling rate in Wavelet object is in milliseconds, otherwise assumed
in seconds. Default: True

\end{itemize}

\end{description}\end{quote}

:param  shot        (int)       Shot number to compute the frequency spectrum. Default 1
:param  fmax:       (float)     Value of the highest frequency expected in the signal. Default: None
:param  plot:       (bool)      Will plot the spectrum and phases of the dataset in the frequency if set to True
\begin{quote}

Default: False
\end{quote}
\begin{quote}\begin{description}
\item[{Parameters}] \leavevmode
\sphinxstyleliteralstrong{\sphinxupquote{fft\_smooth}} \textendash{} (int)       Parameter used to multiply the number of samples inputted into the Fast Fourier
Transform. Increase this factor for a smoother plot. The final number of sample
points will be the nearest power of two of fft\_smooth multiplied by the original
number of time samples in the signal. Higher value increases computational time.
Default: 1

\item[{Returns}] \leavevmode
xf          (np.array)  1D array containing the frequencies

\item[{Returns}] \leavevmode
yf          (np.array)  1D array containing the power of the frequencies in dB

\item[{Returns}] \leavevmode
phase       (np.array)  1D array containing the unwrapped phases at each frequency

\end{description}\end{quote}

\end{fulllineitems}

\index{gausswindow() (in module fullwaveqc.siganalysis)@\spxentry{gausswindow()}\spxextra{in module fullwaveqc.siganalysis}}

\begin{fulllineitems}
\phantomsection\label{\detokenize{index:fullwaveqc.siganalysis.gausswindow}}\pysiglinewithargsret{\sphinxcode{\sphinxupquote{fullwaveqc.siganalysis.}}\sphinxbfcode{\sphinxupquote{gausswindow}}}{\emph{samples}, \emph{wstart}, \emph{wend}, \emph{dt}}{}
Create a gaussian function to window a signal. Standard deviation of the gaussian function is equivalent to one
quarter the window width
\begin{quote}\begin{description}
\item[{Parameters}] \leavevmode\begin{itemize}
\item {} 
\sphinxstyleliteralstrong{\sphinxupquote{samples}} \textendash{} (int) Total of points in the function

\item {} 
\sphinxstyleliteralstrong{\sphinxupquote{wstart}} \textendash{} (int) Start time of the window (ms)

\item {} 
\sphinxstyleliteralstrong{\sphinxupquote{wend}} \textendash{} (int) End time of the window (ms)

\item {} 
\sphinxstyleliteralstrong{\sphinxupquote{dt}} \textendash{} (float) Time sampling of the signal(ms)

\end{itemize}

\item[{Returns}] \leavevmode
w        (numpy.array) 1D array of a gaussian window of size (samples, 1)

\end{description}\end{quote}

\end{fulllineitems}

\index{phasediff() (in module fullwaveqc.siganalysis)@\spxentry{phasediff()}\spxextra{in module fullwaveqc.siganalysis}}

\begin{fulllineitems}
\phantomsection\label{\detokenize{index:fullwaveqc.siganalysis.phasediff}}\pysiglinewithargsret{\sphinxcode{\sphinxupquote{fullwaveqc.siganalysis.}}\sphinxbfcode{\sphinxupquote{phasediff}}}{\emph{PredData}, \emph{ObsData}, \emph{f=1}, \emph{wstart=200}, \emph{wend=1000}, \emph{nr\_min=0}, \emph{nr\_max=None}, \emph{ns\_min=0}, \emph{ns\_max=None}, \emph{ms=True}, \emph{fft\_smooth=3}, \emph{scale=1}, \emph{unwrap=True}, \emph{plot=False}, \emph{verbose=1}}{}
Computes and plots the phase difference between an observed and predicted dataset, at a single specified frequency,
for all receivers and shots. Will present undesired unwrapping effects in the presence of noise or low-amplitude
signal. Calculates phase\_observed - phase\_predicted.
\begin{quote}\begin{description}
\item[{Parameters}] \leavevmode\begin{itemize}
\item {} 
\sphinxstyleliteralstrong{\sphinxupquote{PredData}} \textendash{} (SegyData)  object outputted from fullwaveqc.tools.load function

\item {} 
\sphinxstyleliteralstrong{\sphinxupquote{ObsData}} \textendash{} (SegyData)  object outputted from fullwaveqc.tools.load function. Should have the same time
sampling as PredData

\item {} 
\sphinxstyleliteralstrong{\sphinxupquote{f}} \textendash{} (float)     Frequency in Hz at which the phase difference should be calculated
Default: 1

\item {} 
\sphinxstyleliteralstrong{\sphinxupquote{wstart}} \textendash{} (int)       Time sample to which start the window for the phase difference computation
Default: 200

\item {} 
\sphinxstyleliteralstrong{\sphinxupquote{wend}} \textendash{} (int)       Time sample to which end the window for the phase difference computation.
If negative will take the entire shot window.
Default: 1000

\item {} 
\sphinxstyleliteralstrong{\sphinxupquote{nr\_max}} \textendash{} (int)       Maximum number of receivers to which calculate the phase difference. If None is
given, then number of receivers is inferred from the datasets.
Default: None

\item {} 
\sphinxstyleliteralstrong{\sphinxupquote{ns\_max}} \textendash{} (int)       Maximum number of sources/shots to which calculate the phase difference. If None is
given then number of sources is inferred from the datasets
Default: None

\item {} 
\sphinxstyleliteralstrong{\sphinxupquote{ms}} \textendash{} (bool)      Set to true if sampling rate in Wavelet object is in miliseconds, otherwise assumed
in seconds.
Default: True

\item {} 
\sphinxstyleliteralstrong{\sphinxupquote{fft\_smooth}} \textendash{} (int)       Parameter used to multiply the number of samples inputted into the Fast Fourier
Transform. Increase this factor for a smoother plot. The final number of sample
points will be the nearest power of two of fft\_smooth multiplied by the original
number of time samples in the signal. Higher value increases computational time.
Recommended minimum of 2 for stable calculations
Default: 3

\end{itemize}

\end{description}\end{quote}
\begin{description}
\item[{:param unwrap       (bool)      If set to true will perform phase unwrapping in the receiver domain. For a detailed}] \leavevmode
discussion of how these might affect the phase difference results, refer to the
project report in the Github repository.
Default: True

\end{description}
\begin{quote}\begin{description}
\item[{Parameters}] \leavevmode\begin{itemize}
\item {} 
\sphinxstyleliteralstrong{\sphinxupquote{plot}} \textendash{} (bool)      Will plot the phase difference if set to True
Default: False

\item {} 
\sphinxstyleliteralstrong{\sphinxupquote{verbose}} \textendash{} (bool)      If set to True will verbose the main steps of the function calculation. Default 1

\end{itemize}

\item[{Returns}] \leavevmode
\begin{description}
\item[{phase\_pred  (np.array)  phase\_pred  2D array of size (ns\_max, nr\_max) with the unwrapped phases of the}] \leavevmode
predicted dataset at the specified frequency

\item[{phase\_obs   (np.array)  phase\_obs  2D array of size (ns\_max, nr\_max) with the unwrapped phases of the}] \leavevmode
observed dataset at the specified frequency

\item[{phase\_diff  (np.array)  phase\_diff  2D array of size (ns\_max, nr\_max) with the unwrapped phase differences}] \leavevmode
between the observed and predicted datasets at the specified frequency

\end{description}


\end{description}\end{quote}

\end{fulllineitems}

\index{wavespec() (in module fullwaveqc.siganalysis)@\spxentry{wavespec()}\spxextra{in module fullwaveqc.siganalysis}}

\begin{fulllineitems}
\phantomsection\label{\detokenize{index:fullwaveqc.siganalysis.wavespec}}\pysiglinewithargsret{\sphinxcode{\sphinxupquote{fullwaveqc.siganalysis.}}\sphinxbfcode{\sphinxupquote{wavespec}}}{\emph{Wavelet}, \emph{ms=True}, \emph{fmax=None}, \emph{plot=False}, \emph{fft\_smooth=1}}{}
Returns and plots the frequency spectrum of a source wavelet.
\begin{quote}\begin{description}
\item[{Parameters}] \leavevmode\begin{itemize}
\item {} 
\sphinxstyleliteralstrong{\sphinxupquote{Wavelet}} \textendash{} (SegyData)   object outputted from fullwaveqc.tools.load function

\item {} 
\sphinxstyleliteralstrong{\sphinxupquote{ms}} \textendash{} (bool)       Set to true if sampling rate in Wavelet object is in miliseconds, otherwise assumed
in seconds. Default: True

\item {} 
\sphinxstyleliteralstrong{\sphinxupquote{fmax}} \textendash{} (float)      Value of the highest frequency expected in the signal
Default: None

\item {} 
\sphinxstyleliteralstrong{\sphinxupquote{plot}} \textendash{} (bool)       Will plot the wavelet in time and wave frequencies if set to True
Default: False

\item {} 
\sphinxstyleliteralstrong{\sphinxupquote{fft\_smooth}} \textendash{} (int)        Parameter used to multiply the number of samples inputted into the Fast Fourier
Transform. Increase this factor for a smoother plot. The final number of sample
points will be the nearest power of two of fft\_smooth multiplied by the original
number of time samples in the signal. Higher value increases computational time.
Default: 1

\end{itemize}

\item[{Returns}] \leavevmode
xf          (np.array)   1D array containing the frequencies

\item[{Returns}] \leavevmode
yf          (np.array)   1D array containing the power of the frequencies in dB

\item[{Returns}] \leavevmode
phase       (np.array)   1D array containing the unwrapped phases at each frequency

\end{description}\end{quote}

\end{fulllineitems}

\index{xcorr() (in module fullwaveqc.siganalysis)@\spxentry{xcorr()}\spxextra{in module fullwaveqc.siganalysis}}

\begin{fulllineitems}
\phantomsection\label{\detokenize{index:fullwaveqc.siganalysis.xcorr}}\pysiglinewithargsret{\sphinxcode{\sphinxupquote{fullwaveqc.siganalysis.}}\sphinxbfcode{\sphinxupquote{xcorr}}}{\emph{PredData}, \emph{ObsData}, \emph{wstart=0}, \emph{wend=-1}, \emph{nr\_min=0}, \emph{nr\_max=None}, \emph{ns\_min=0}, \emph{ns\_max=None}, \emph{ms=True}, \emph{plot=False}, \emph{verbose=1}}{}
Computes and plots the cross-correlation between an observed and predicted dataset using numpy.correlate.
Traces are normalised to unit length for comparison
\begin{quote}\begin{description}
\item[{Parameters}] \leavevmode\begin{itemize}
\item {} 
\sphinxstyleliteralstrong{\sphinxupquote{PredData}} \textendash{} (SegyData) object outputted from fullwaveqc.tools.load function

\item {} 
\sphinxstyleliteralstrong{\sphinxupquote{ObsData}} \textendash{} (SegyData) object outputted from fullwaveqc.tools.load function

\item {} 
\sphinxstyleliteralstrong{\sphinxupquote{wstart}} \textendash{} (int)      Time sample to which start the window for the phase difference computation
Default: 200

\item {} 
\sphinxstyleliteralstrong{\sphinxupquote{wend}} \textendash{} (int)      Time sample to which end the window for the phase difference computation.
If negative will use the entire shot window.
Default: 1000

\item {} 
\sphinxstyleliteralstrong{\sphinxupquote{nr\_max}} \textendash{} (int)      Maximum number of receivers to which calculate the phase difference. If None is given
then max number of receivers is inferred from the PredData and ObsData
Default: None

\item {} 
\sphinxstyleliteralstrong{\sphinxupquote{ns\_max}} \textendash{} (int)      Maximum number of sources/shots to which calculate the phase difference. If None is
given, then max number of sources is inferred from the datasets
Default: None

\item {} 
\sphinxstyleliteralstrong{\sphinxupquote{ms}} \textendash{} (bool)     Set to true if sampling rate in Wavelet object is in miliseconds, otherwise assumed
in seconds.
Default: True

\item {} 
\sphinxstyleliteralstrong{\sphinxupquote{plot}} \textendash{} (bool)     Will plot the phase difference if set to True
Default: False

\item {} 
\sphinxstyleliteralstrong{\sphinxupquote{verbose}} \textendash{} (bool)     If set to True will verbose the main steps of the function calculation. Default 1

\end{itemize}

\item[{Returns}] \leavevmode
\begin{description}
\item[{xcorr\_arr   (np.array) 2D array of size (ns\_max, nr\_max) with the unwrapped phases of the}] \leavevmode
predicted dataset at the specified frequency

\item[{(np.array)  phase\_obs  2D array of size (ns\_max, nr\_max) with the unwrapped phases of the}] \leavevmode
observed dataset at the specified frequency

\item[{(np.array)  phase\_diff  2D array of size (ns\_max, nr\_max) with the unwrapped phase differences}] \leavevmode
between the observed and predicted datasets at the specified frequency

\end{description}


\end{description}\end{quote}

\end{fulllineitems}

\phantomsection\label{\detokenize{index:module-fullwaveqc.inversion}}\index{fullwaveqc.inversion (module)@\spxentry{fullwaveqc.inversion}\spxextra{module}}\index{functional() (in module fullwaveqc.inversion)@\spxentry{functional()}\spxextra{in module fullwaveqc.inversion}}

\begin{fulllineitems}
\phantomsection\label{\detokenize{index:fullwaveqc.inversion.functional}}\pysiglinewithargsret{\sphinxcode{\sphinxupquote{fullwaveqc.inversion.}}\sphinxbfcode{\sphinxupquote{functional}}}{\emph{filepath}, \emph{name=None}, \emph{plot=False}}{}
Retrieves and plots the global functional value of an inversion run with Fullwave3D as  recorded in the job log
file
\begin{quote}\begin{description}
\item[{Parameters}] \leavevmode\begin{itemize}
\item {} 
\sphinxstyleliteralstrong{\sphinxupquote{filepath}} \textendash{} (str)      path to \textless{}PROJECT\_NAME\textgreater{}\_job,log file

\item {} 
\sphinxstyleliteralstrong{\sphinxupquote{name}} \textendash{} (str)      name of the project. If not given, will be inferred from filename. Default: None

\item {} 
\sphinxstyleliteralstrong{\sphinxupquote{plot}} \textendash{} (bool)     if true will produce a plot of the functional values vs iterations. Default: False

\end{itemize}

\item[{Returns}] \leavevmode
iter\_all     (np.array) array of iteration numbers of the run of size (number\_of\_iterations, 1)

\item[{Returns}] \leavevmode
func         (np.array) array of functional of size (number\_of\_iterations, 1)

\end{description}\end{quote}

\end{fulllineitems}

\index{steplen() (in module fullwaveqc.inversion)@\spxentry{steplen()}\spxextra{in module fullwaveqc.inversion}}

\begin{fulllineitems}
\phantomsection\label{\detokenize{index:fullwaveqc.inversion.steplen}}\pysiglinewithargsret{\sphinxcode{\sphinxupquote{fullwaveqc.inversion.}}\sphinxbfcode{\sphinxupquote{steplen}}}{\emph{filepath}, \emph{name=None}, \emph{plot=False}}{}
Retrieves and plots the step length value of an inversion run with Fullwave3D as recorded in the job log
file
\begin{quote}\begin{description}
\item[{Parameters}] \leavevmode\begin{itemize}
\item {} 
\sphinxstyleliteralstrong{\sphinxupquote{filepath}} \textendash{} (str)      path to \textless{}PROJECT\_NAME\textgreater{}\_job,log file

\item {} 
\sphinxstyleliteralstrong{\sphinxupquote{name}} \textendash{} (str)      name of the project. If not given, will be inferred from filename. Default: None

\item {} 
\sphinxstyleliteralstrong{\sphinxupquote{plot}} \textendash{} (bool)     if true will produce a plot of the step length values vs iterations. And highlight
in red all negative step length values. Default: False

\end{itemize}

\item[{Returns}] \leavevmode
iter\_all     (np.array) array of iteration numbers of the run of size (number\_of\_iterations, 1)

\item[{Returns}] \leavevmode
steplenarr   (np.array) array of step length values of size (number\_of\_iterations, 1)

\end{description}\end{quote}

\end{fulllineitems}



\chapter{Indices and tables}
\label{\detokenize{index:indices-and-tables}}\begin{itemize}
\item {} 
\DUrole{xref,std,std-ref}{genindex}

\item {} 
\DUrole{xref,std,std-ref}{modindex}

\item {} 
\DUrole{xref,std,std-ref}{search}

\end{itemize}


\renewcommand{\indexname}{Python Module Index}
\begin{sphinxtheindex}
\let\bigletter\sphinxstyleindexlettergroup
\bigletter{f}
\item\relax\sphinxstyleindexentry{fullwaveqc.geom}\sphinxstyleindexpageref{index:\detokenize{module-fullwaveqc.geom}}
\item\relax\sphinxstyleindexentry{fullwaveqc.inversion}\sphinxstyleindexpageref{index:\detokenize{module-fullwaveqc.inversion}}
\item\relax\sphinxstyleindexentry{fullwaveqc.siganalysis}\sphinxstyleindexpageref{index:\detokenize{module-fullwaveqc.siganalysis}}
\item\relax\sphinxstyleindexentry{fullwaveqc.tools}\sphinxstyleindexpageref{index:\detokenize{module-fullwaveqc.tools}}
\item\relax\sphinxstyleindexentry{fullwaveqc.visual}\sphinxstyleindexpageref{index:\detokenize{module-fullwaveqc.visual}}
\end{sphinxtheindex}

\renewcommand{\indexname}{Index}
\printindex
\end{document}